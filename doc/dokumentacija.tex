\documentclass[12pt,fleqn]{article}
\usepackage{polyglossia}
\usepackage{fontspec}
\setmainlanguage{serbian}
\newfontfamily\cyrillicfont{PT Sans}[Script=Cyrillic]
\newfontfamily\cyrillicfonttt{PT Sans}[Script=Cyrillic]
\newfontfamily\cyrillicfontsf{PT Sans}[Script=Cyrillic]
\usepackage{authblk}
\usepackage{blindtext}


\usepackage[style=numeric,backend=biber]{biblatex}
\addbibresource{literatura.bib}

\title{Проблем ротације усева}
\author{Никола Шутић}
\affil{Математички факултет, Универзитет у Београду}

\begin{document}
\maketitle
\newpage

\tableofcontents
\newpage

\section{Увод}
Традиционално гајењо воће и поврће подразумева вишегодишњу поновљену садњу монокултуре (једне исте биљне врсте) на истом земљишту. Оваква стратегија садње има предност у једноставности планирања и извођења, погодног за механизацију и садње на велико. Међутим, оваква стратегија може изазвати озбиљне проблеме и спречити дугорочну експлотацију земље. Проблеми који се на овај начин могу да се изазову су, између осталог: исушивање хранљивих супстанци земље, прекомерно размножавање патогена, смањени принос.

С обзиром на напор неких научника и доктора да свет погурају више у смеру здравије исхране и помере од високо обрађене хране, вреди даље истражити и улагати у технике које би помогле и малим газдинствима да произведу веће количине воћа и поврћа за своје потребе.

\section{Проблем ротације усева}
\subsection{Дефиница проблема}
Уколико на истом земљишту смењујемо биљке које садимо на годишњем, сезонском или неком другом временском интервалу, ми вршимо ротацију усева. Проблемом ротације усева мисли се на његово решење. Решење овог проблема је управо један план садне који описује низ култура и њихових планираних временских интервала за сетву. Описан је концепт ротације усева, али не и шта желимо да добијемо ротацијом. Најчешћи циљ је постизање највећег могућег приноса од одабраних култура. Неки други циљеви које ће бити обрађени у раду су разноврсност или продаја.

\section{Имплементација алгоритма}
\subsection{Генетски алгоритам}
Генетски алгоритам је врста оптимизационог алгоритма чији се рад заснива на генетичким принципма виђеним у природи. Кандидати за решење проблема се називају хромозоми, и они су најчешће дати као математичка репрезентација решења проблема који решавамо. Користећи операторе укрштања и мутације прилагођене проблему, усмеравамо популацију ка бољем прилагођавању захтевима проблема. Као решење проблема управо узимамо најбоље прилагођену јединку из популације.

Сам генетски алгоритам не захтева велико умеће за имплементацију и примену. Највећа тежине ове техника лежи у успешном математичком моделирању проблема, и прилагођавању оператора алгоритма проблему који се решава. У складу с овиме, временом су се издвојили принципи за моделирање проблема који су се често понављали. У овом раду биће коришћено 0-1 бинарно моделиране проблема.

\subsection{Ротација усева као 0-1 ГА}

У овом одељку ће бити детаљније објашњено приступ моделирања проблема коришћен у овом раду. Ослањајући се на претходни рад \cite{geraldi}, уводе се следеће променљиве које ће бити коришћене:

$x_{ijk} \in \{0,1\}$ - Један ген у плану, вредност обележава да ли је биљка $i$, засађена у периоду $j$, на земљишту $k$.

$N$ - Укупан број биљака узетих у разматрање.

$M$ - Број временских периода. Један период узет да буде 15 дана али може бити више или мање по потреби.

$B$ - Скуп скупова биљака одвојене по породицама.

$I_i$ - Временски периоди за сетву биљке $i$.

$D$ - Скуп биљака које се користе за зелено ђубриво.

\subsection{Мека и тврда ограничења}
За разлику од \cite{geraldi}, ограничења претходно коришћена за потребе ограничења, овде ће бити имплементиране и коришћене у виду пенала јединке и то: $\sigma_{soft}$ и $\sigma_{hard}$. Огранићења коришђена су:


\begin{enumerate}
  
\item{\[
  \sum_{i=o}^n \sum_{q=0}^{t_i-1} x_{i(j-q)k} \le 1 \qquad j = 1..M,\, k = 1..K
\] На истој земљи у једно време може расти само једна биљка.}

\item{
    \[
      \sum_{i\in D} \sum_{j\in I_i}x_{ijk} \ge 1 \qquad k = 1..K
    \] Свако земљиште мора бити барем једном ђубрено.
  }

\end{enumerate}

\subsection{Главна петља}
\blindtext

\subsection{Исцрпни алгоритам}
\blindtext

\section{Тестирање и резултати}
\subsection{Исцрпни алгоритам}
\blindtext

\subsection{Генетски алгоритам}
\blindtext

\printbibliography

\end{document}
